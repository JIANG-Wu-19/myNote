\documentclass[UTF8,12pt]{article}
\usepackage{ctex}
\usepackage{indentfirst}
\usepackage{color}
\usepackage{hyperref}
\usepackage{graphicx}
\usepackage{subfigure}
\usepackage{pdfpages}
\usepackage{listings}
\usepackage{afterpage}
\usepackage{geometry}
\usepackage{booktabs}
\usepackage{multirow}
\usepackage{graphicx}

\geometry{a4paper,scale=0.8}

\newcommand\myemptypage{
    \null
    \thispagestyle{empty}
    \addtocounter{page}{-1}
    \newpage
}

\hypersetup{
    hidelinks,
	colorlinks=true,
	allcolors=black,
	pdfstartview=Fit,
	breaklinks=true
}

\definecolor{dkgreen}{rgb}{0,0.6,0}
\definecolor{gray}{rgb}{0.5,0.5,0.5}
\definecolor{mauve}{rgb}{0.58,0,0.82}

\lstset{ %
  language=Octave,                % the language of the code
  basicstyle=\footnotesize,           % the size of the fonts that are used for the code
  numbers=left,                   % where to put the line-numbers
  numberstyle=\tiny\color{gray},  % the style that is used for the line-numbers
  stepnumber=2,                   % the step between two line-numbers. If it's 1, each line 
                                  % will be numbered
  numbersep=5pt,                  % how far the line-numbers are from the code
  backgroundcolor=\color{white},      % choose the background color. You must add \usepackage{color}
  showspaces=false,               % show spaces adding particular underscores
  showstringspaces=false,         % underline spaces within strings
  showtabs=false,                 % show tabs within strings adding particular underscores
  frame=single,                   % adds a frame around the code
  rulecolor=\color{black},        % if not set, the frame-color may be changed on line-breaks within not-black text (e.g. commens (green here))
  tabsize=2,                      % sets default tabsize to 2 spaces
  captionpos=b,                   % sets the caption-position to bottom
  breaklines=true,                % sets automatic line breaking
  breakatwhitespace=false,        % sets if automatic breaks should only happen at whitespace
  title=\lstname,                   % show the filename of files included with \lstinputlisting;
                                  % also try caption instead of title
  keywordstyle=\color{blue},          % keyword style
  commentstyle=\color{dkgreen},       % comment style
  stringstyle=\color{mauve},         % string literal style
  escapeinside={\%*}{*)},            % if you want to add LaTeX within your code
  morekeywords={*,...}               % if you want to add more keywords to the set
}


\setlength{\parindent}{2em}

\begin{document}

\begin{titlepage}
    \includepdf[pages={1}]{cover3.pdf}
\end{titlepage}

\myemptypage

\begin{center}
    \tableofcontents
\end{center}

\newpage

% 一、实验目的
% 指出此次实验应该达到的学习目标。
% 二、实验内容
% 指出此次实验应完成的任务。
% 三、实验方法
% 包括实验方法、原理、技术、方案等。
% 四、实验步骤
% 指出完成该实验的操作步骤。
% 五、实验结果
% 记录实验输出数据和结果。
% 六、实验结论
% 对实验数据和结果进行分析描述,给出实验取得的成果和结论。
% 注:有程序的要求附上程序源代码,有图表的要有截图并有相应的文字说明和分析
% 七、实验小结
% 给出本次实验的体会,如学会了什么,遇到哪些问题,如何解决这些问题,存在哪些有待改进的地方。

实验学时:2

每组人数:3

实验类别:2\ (1:基础性\ 2:综合性\ 3:设计性\ 4:研究性)

实验要求:1\ (1:必修\ 2:选修\ 3:其它)

实验类别:3\ (1:基础\ 2:专业基础\ 3:专业\ 4:其它)

\section{实验目的}


\section{实验内容}


\section{实验方法}
\subsection{实验原理}


\subsection{试验方案及调试过程}


\section{实验步骤}



\section{实验结果}

\section{实验结论}


\section{实验小结}


\newpage

\section{实验3源代码}
\begin{lstlisting}[frame=shadowbox]
/**
  主程序:系统初始化后使用串口通信对Timer3的自动重载寄存器(ARR)初值进行选择,通过配置预分频,
	将Timer3的时钟频率设置为10KHz,在配置好定时器后开启定时器中断。Timer3的计数器寄存器从0开始以10KHz的频率递增,
	当其值大于ARR寄存器中的数值时,产生上溢,即定时器中断。
	注意:此例程中,每产生一次定时器中断,LED发生一次跳变,即模式1中LED闪烁频率为0.5Hz,模式2中LED闪烁频率为5Hz
**/
#include "main.h"
#include "system_init.h"

#define TIMEOUT   10000       // 10 seconds

/* Prescaler declaration */
uint8_t uRxBuffer = '1';			// 默认选择模式"1. 10000"
uint16_t ARRValue = 10000 - 1;// 默认ARR寄存器值为10000 - 1
extern UART_HandleTypeDef   UartHandle;
extern TIM_HandleTypeDef    TimHandle;

void System_Init(void);
void Timer_Config(uint16_t ARRValue);
    
/* main */
int main(void)
{
  System_Init();
  
  // printf("\n\r************************************\n\r");
  // printf("* Timer counter frequency is 10KHz *\n\r* Upcounting mode                  *\n\r* Initial value is 0               *");
  // printf("\n\r************************************\n\r");
  // printf("\n\rSelect the value of ARR register:\n\r1. 10000(default)    2. 1000\n\r");
  
  // /* Receive data from UART */
  // HAL_UART_Receive(&UartHandle, (uint8_t*)&uRxBuffer, 1, TIMEOUT);
  
  // if (uRxBuffer == '2')
  //   ARRValue = 1000 - 1;

  // Timer_Config(ARRValue);	//使用Tim3在这里面配置中断

  // /* Start the TIM Base generation in interrupt mode */
  // HAL_TIM_Base_Start_IT(&TimHandle);
  
  // printf("\n\rExample finished\n\r");
  
  while (1)
  {
    uint8_t stack[100];
    int store=0;

    double num=0;
    printf("\n\r**** UART-Timer ****\n\rPlease enter the frequency...\n\r");

    //receive data from uart
    while(1){
      HAL_UART_Receive(&UartHandle,(uint8_t*)&uRxBuffer,1,TIMEOUT);
      printf("%c",uRxBuffer);
      if(uRxBuffer!='\r'&&uRxBuffer!='\n'){
        if((uRxBuffer>='0'&&uRxBuffer<='9')||uRxBuffer=='.'){
          stack[store]=uRxBuffer;
          store++;
        }else{
          printf("\n\r INPUT ERROR! \n\r");
          break;
        }
      }

      if(uRxBuffer=='\r'||uRxBuffer=='\n'||store>10){
        sscanf((const char*)stack,"%lf",&num);
        ARRValue=5000/num-1;
        break;
      }
    }

    Timer_Config(ARRValue);

    HAL_TIM_Base_Start_IT(&TimHandle);
  }
}

void HAL_TIM_PeriodElapsedCallback(TIM_HandleTypeDef *htim)
{
  BSP_LED_Toggle(LED1);
}
\end{lstlisting}

\end{document}
